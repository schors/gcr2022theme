% !TeX TS-program = xelatex

\documentclass[aspectratio=169]{beamer}

\usepackage{xltxtra}
\usepackage[main=russian,english]{babel}
\defaultfontfeatures{Mapping=tex-text}
\usepackage{listings}

\usepackage[backend=biber,bibstyle=numeric,citestyle=numeric-comp,sorting=none,defernumbers=true,date=long]{biblatex}
\addbibresource{gcr2022-example.bib}

\usetheme{gcr2022}

% for biblatex
\setbeamerfont{bibliography item}{size*={8pt}{1}}
\setbeamertemplate{bibliography item}{\insertbiblabel}
\renewcommand*{\bibfont}{\sffamily\fontsize{8}{8}\selectfont}
\DeclareFieldFormat{url}{\color{blue}\url{#1}}

% for listings
\definecolor{goplaybg}{HTML}{FFFFDD}
\definecolor{goplayc}{HTML}{202224}
\definecolor{goplayl}{HTML}{00add8}
\definecolor{goplayh}{HTML}{EFEF9C}
\lstset{
        columns=flexible,
        keepspaces=true,
        showstringspaces=false,
        showtabs=false,
        tabsize=4,
        frame=single,
        basicstyle=\fontsize{10pt}{12}\ttfamily\color{goplayc},
        backgroundcolor=\color{goplaybg},
        commentstyle=\color{black},
        keywordstyle=\color{black},
        stringstyle=\color{red},
        rulecolor=\color{goplayl},
        framerule=1pt
}

% for listings background
\makeatletter
\let\old@lstKV@SwitchCases\lstKV@SwitchCases
\def\lstKV@SwitchCases#1#2#3{}
\makeatother
\usepackage{lstlinebgrd}
\makeatletter
\let\lstKV@SwitchCases\old@lstKV@SwitchCases

\lst@Key{numbers}{none}{%
    \def\lst@PlaceNumber{\lst@linebgrd}%
    \lstKV@SwitchCases{#1}%
    {none:\\%
     left:\def\lst@PlaceNumber{\llap{\normalfont
                \lst@numberstyle{\thelstnumber}\kern\lst@numbersep}\lst@linebgrd}\\%
     right:\def\lst@PlaceNumber{\rlap{\normalfont
                \kern\linewidth \kern\lst@numbersep
                \lst@numberstyle{\thelstnumber}}\lst@linebgrd}%
    }{\PackageError{Listings}{Numbers #1 unknown}\@ehc}}
\makeatother

\makeatletter
%%%%%%%%%%%%%%%%%%%%%%%%%%%%%%%%%%%%%%%%%%%%%%%%%%%%%%%%%%%%%%%%%%%%%%%%%%%%%%
%
% \btIfInRange{number}{range list}{TRUE}{FALSE}
%
% Test in int number <number> is element of a (comma separated) list of ranges
% (such as: {1,3-5,7,10-12,14}) and processes <TRUE> or <FALSE> respectively

\newcount\bt@rangea
\newcount\bt@rangeb

\newcommand\btIfInRange[2]{%
    \global\let\bt@inrange\@secondoftwo%
    \edef\bt@rangelist{#2}%
    \foreach \range in \bt@rangelist {%
        \afterassignment\bt@getrangeb%
        \bt@rangea=0\range\relax%
        \pgfmathtruncatemacro\result{ ( #1 >= \bt@rangea) && (#1 <= \bt@rangeb) }%
        \ifnum\result=1\relax%
            \breakforeach%
            \global\let\bt@inrange\@firstoftwo%
        \fi%
    }%
    \bt@inrange%
}
\newcommand\bt@getrangeb{%
    \@ifnextchar\relax%
        {\bt@rangeb=\bt@rangea}%
        {\@getrangeb}%
}
\def\@getrangeb-#1\relax{%
    \ifx\relax#1\relax%
        \bt@rangeb=100000%   \maxdimen is too large for pgfmath
    \else%
        \bt@rangeb=#1\relax%
    \fi%
}

%%%%%%%%%%%%%%%%%%%%%%%%%%%%%%%%%%%%%%%%%%%%%%%%%%%%%%%%%%%%%%%%%%%%%%%%%%%%%%
%
% \btLstHL<overlay spec>{range list}
%
% TODO BUG: \btLstHL commands can not yet be accumulated if more than one overlay spec match.
% 
\newcommand<>{\btLstHL}[1]{%
\only#2{\btIfInRange{\value{lstnumber}}{#1}{\color{goplayh}\def\lst@linebgrdcmd{\color@block}}{\def\lst@linebgrdcmd####1####2####3{}}}%
}%
\makeatother

% references example begin
%\usepackage{filecontents}
%\begin{filecontents}{\jobname.bib}
%@online{oleafbeamer,
%        title   = "Beamer - Overleaf, Online LaTeX Editor",
%        url     = "https://www.overleaf.com/learn/latex/Beamer"
%}
%@online{present-code,
%        author  = "Uri Nativ",
%        title   = "How to present code.",
%        year    = "2016",
%        month   = "4",
%        url     = "https://www.slideshare.net/LookAtMySlides/codeware"
%}
%\end{filecontents}
%\bibliography{\jobname}
% references example end
% but in real life you may use an external .bib file and \addbibresource{shl2021-example.bib}
% be careful tha you have to run xelatex, biber, xelatex, xelatex (twice for example)

\title{Шаблон презентации\\HighLoad++ 2022 Foundation\\секция GolangConf}
\author{Олег Бунин}

\begin{document}

\begin{frame}
\titlepage
\end{frame}

\begin{frame}{Постановка проблемы}
        Начните, пожалуйста, с \textbf{постановки проблемы}. Какую задачу вы решали? Почему эта задача актуальна и важна не только для вас?\\
        ~\\
        Почему вы думаете, что это проблема, в принципе?\\
        ~\\
        Не лишним будет привести числовые оценки: объёмы данных, необходимая скорость и так далее.
\end{frame}

\begin{frame}{Варианты решений}
        Какие существуют решения вашей проблемы? Даже если вы выбрали решение, в двух словах, расскажите, пожалуйста, обо всех вариантах. \supercite{oleafbeamer}\\ 
        ~\\
        По \textbf{каким критериям} и как вообще подойти к выбору? Что надо сравнивать?
\end{frame}

\begin{frame}{Варианты решений}
        Почему вы выбрали именно то решение, что выбрали? На что надо обратить внимание вашим слушателям, чтобы сделать правильный выбор.\\
        ~\\
        Также важны те решения, которые вы пробовали, но они \textbf{не получились} или не получились с первого раза. \supercite{present-code}
\end{frame}

\begin{frame}{Описание решения}
        \begin{itemize}
                \item Вот только теперь мы описываем решение, которое привело вас к результату.
                \item Здесь важны упоминания кейсов, любое решение мы описываем максимально конкретно, мы ведь \textbf{прикладная} конференция.
                \item И не надо миндальничать – аудитория конференции вполне профессиональна, жалеть её не нужно
        \end{itemize}
\end{frame}

\begin{frame}{Выводы и рекомендации}
        Наша задача не только рассказать о решении, но и облегчить задачу внедрения.\\
        ~\\
        Какие выводы вы могли бы сделать?\\
        ~\\
        С чего начать слушателям? Допустим, им нравится ваше решение, \textbf{какой их следующий шаг}? \supercite{present-latex}
\end{frame}

\begin{frame}[fragile]{Пример кода}
	\begin{lstlisting}[linewidth=0.9\textwidth,linebackgroundcolor={%
                                                        \btLstHL<1>{3}%
                                                        \btLstHL<2>{6}%
                                                        }]
package main

import "fmt"

func main() {
        fmt.Println("Hello, мир!")
}       \end{lstlisting}
\end{frame}

\nocite{*}
\setbeamertemplate{frametitle continuation}{}
\begin{frame}[t]{Ссылки}
\printbibliography
\end{frame}

\end{document}


