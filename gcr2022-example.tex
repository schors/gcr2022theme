% !TeX TS-program = xelatex

\documentclass[aspectratio=169]{beamer}

\usepackage{xltxtra}
\usepackage[main=russian,english]{babel}
\defaultfontfeatures{Mapping=tex-text}

% Chinese support
\usepackage{xeCJK}

\usetheme{gcr2022}

% add bib -file
\addbibresource{gcr2022-example.bib}

\title{Шаблон презентации\\HighLoad++ 2022 Foundation\\секция GolangConf}
\author{Олег Бунин}

\begin{document}

\begin{frame}
\titlepage
\end{frame}

\begin{frame}{Постановка проблемы}
        Начните, пожалуйста, с \textbf{постановки проблемы}. Какую задачу вы решали? Почему эта задача актуальна и важна не только для вас?\\
        ~\\
        Почему вы думаете, что это проблема, в принципе?\\
        ~\\
        Не лишним будет привести числовые оценки: объёмы данных, необходимая скорость и так далее.
\end{frame}

\begin{frame}{Варианты решений}
        Какие существуют решения вашей проблемы? Даже если вы выбрали решение, в двух словах, расскажите, пожалуйста, обо всех вариантах. \supercite{oleafbeamer}\\ 
        ~\\
        По \textbf{каким критериям} и как вообще подойти к выбору? Что надо сравнивать?
\end{frame}

\begin{frame}{Варианты решений}
        Почему вы выбрали именно то решение, что выбрали? На что надо обратить внимание вашим слушателям, чтобы сделать правильный выбор.\\
        ~\\
        Также важны те решения, которые вы пробовали, но они \textbf{не получились} или не получились с первого раза. \supercite{present-code}
\end{frame}

\begin{frame}{Описание решения}
        \begin{itemize}
                \item Вот только теперь мы описываем решение, которое привело вас к результату.
                \item Здесь важны упоминания кейсов, любое решение мы описываем максимально конкретно, мы ведь \textbf{прикладная} конференция.
                \item И не надо миндальничать – аудитория конференции вполне профессиональна, жалеть её не нужно
        \end{itemize}
\end{frame}

\begin{frame}{Выводы и рекомендации}
        Наша задача не только рассказать о решении, но и облегчить задачу внедрения.\\
        ~\\
        Какие выводы вы могли бы сделать?\\
        ~\\
        С чего начать слушателям? Допустим, им нравится ваше решение, \textbf{какой их следующий шаг}? \supercite{present-latex}
\end{frame}

\begin{frame}[fragile]{Пример кода}
	\begin{lstlisting}[linewidth=0.9\textwidth,linebackgroundcolor={%
                                                        \btLstHL<1>{3}%
                                                        \btLstHL<2>{6}%
                                                        }]
package main

import "fmt"

func main() {
        fmt.Println("Привет, 世界!")
}       \end{lstlisting}
\end{frame}

\nocite{*}
\setbeamertemplate{frametitle continuation}{}
\begin{frame}[t]{Ссылки}
\printbibliography
\end{frame}

\end{document}


